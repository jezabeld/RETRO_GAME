% Created 2025-07-09
% Intended LaTeX compiler: pdflatex
\documentclass[11pt,a4paper]{article}

% ──────────────────────── Paquetes básicos ────────────────────────
\usepackage[utf8]{inputenc}
\usepackage[T1]{fontenc}
\usepackage[english, spanish]{babel}
\usepackage{graphicx}
\usepackage{grffile}
\usepackage{longtable}
\usepackage{wrapfig}
\usepackage{rotating}
\usepackage[normalem]{ulem}
\usepackage{amsmath, amssymb}
\usepackage{textcomp}
\usepackage{capt-of}
\usepackage{tabularx}
\usepackage{lastpage}
\usepackage{enumitem}
\usepackage[table,xcdraw]{xcolor}
\usepackage[left=2.00cm,right=2.50cm,top=2.50cm,bottom=2.50cm]{geometry}
\usepackage{hyperref}

% ──────────────────────── Comandos útiles ────────────────────────
\newcommand*{\autor}[1]{\def\authorname{#1}}
\newcommand*{\titulo}[1]{\def\@title{#1}\def\ttitle{#1}}
\newcommand{\versionActual}{A}
\newcommand{\fechaA}{09/07/2025}
\newcommand{\docCode}{\normalsize RETRO\_GAME-DD versión \versionActual}
\newcommand{\fechaActual}{\fechaA}

\titulo{Videojuego portátil inspirado en consolas retro}      % <─ Cambia si quieres
\autor{Lic. Jezabel Danon}

% ──────────────────────── Encabezado / pie ────────────────────────
\usepackage{fancyhdr}
\fancyhf{}
\pagestyle{fancy}
\lhead{\includegraphics[width=3.5cm]{./Figuras/logoFIUBA.pdf}}
\rhead{\normalsize\textbf{\@title}\\Documento de diseño de software\\\docCode}
\setlength{\headheight}{42pt}
\setlength{\footskip}{25pt}
\cfoot{\normalsize Página \thepage\ de \pageref{LastPage}}
\renewcommand{\headrulewidth}{1pt}
\renewcommand{\footrulewidth}{0.4pt}

% ──────────────────────── Hyperref ────────────────────────
\hypersetup{
  colorlinks=true,
  linkcolor=black,
  urlcolor=blue,
  pdftitle={\@title},
  pdfauthor={\authorname},
  pdflang={es}
}

% ──────────────────────── Documento ────────────────────────
\begin{document}

% ---------- Portada ----------
\begin{titlepage}
  \centering
  \includegraphics[width=.7\textwidth]{Figuras/logoFIUBA.pdf}\par
  \vspace{1cm}
  {\Huge\textbf{\ttitle}}\par
  \vspace{1.5cm}
  {\Large\itshape Documento de diseño de software (SDD)\par}
  \vspace{3cm}
  \flushleft
  {\normalsize Autor:}\par
  {\Large \authorname\ (jezabel.danon@gmail.com)}\par
  \vspace{1.5cm}
  {\scshape\LARGE \fechaActual}\par
  {\scshape\LARGE Versión \versionActual}\par
  \vfill
  \centering
  \textit{Basado en el estándar IEEE 1016-2017 e integrado con la ERS RETRO\_GAME-RS.}
\end{titlepage}

\clearpage
\tableofcontents
\clearpage

% ---------- 1 Introducción ----------
\section{Introducción}
\subsection{Propósito}
Breve descripción de por qué existe este documento, a quién va dirigido y cómo se relaciona con la ERS.

\subsection{Ámbito}
Alcance del diseño: qué módulos, capas o servicios cubre; qué queda fuera.

\subsection{Definiciones y acrónimos}
Si algún término de diseño no aparece ya en la ERS, defínelo aquí.

% ---------- 2 Visión general de la arquitectura ----------
\section{Visión general de la arquitectura}
\subsection{Vista en capas}
\begin{itemize}
  \item Reproduce (o amplía) el diagrama \textit{SW\_layers.pdf}, destacando los límites entre
        Bootloader, Drivers, Lógica del sistema y Lógica del juego.
  \item Explica brevemente las dependencias entre capas.
\end{itemize}

\subsection{Estilos y patrones arquitectónicos}
Enumera decisiones globales—RTOS cooperativo, uso de colas, pub/sub, etc.

% ---------- 3 Vista de módulos / servicios ----------
\section{Descripción de módulos y servicios}
Para cada servicio descrito en el glosario:
\subsection{Boot Manager}
\begin{itemize}
  \item \textbf{Responsabilidad:} \ldots
  \item \textbf{Interfaces:} API C (\texttt{void Boot\_Init(void);})\ldots
  \item \textbf{Datos persistentes:} —
  \item \textbf{Errores y manejo:} CRC fallido → LED parpadea\ldots
\end{itemize}

\subsection{UI Controller}
\ldots (repite la misma plantilla para cada servicio: Resource Loader, Event Dispatcher, Save Manager, FSM-Juego, etc.)

% ---------- 4 Modelos dinámicos ----------
\section{Modelos dinámicos}
\subsection{Diagrama de secuencia 1 - Encendido hasta menú}
% \begin{center}
% %   \includegraphics[width=.9\linewidth]{Figuras/seq_boot_to_menu.pdf}
% \end{center}
Descripción narrativa de cada paso.

\subsection{Diagrama de secuencia 2 - Pausa y opciones}
% \begin{center}
%   \includegraphics[width=.9\linewidth]{Figuras/seq_pause_menu.pdf}
% \end{center}

% ---------- 5 Asignación a tareas y concurrencia ----------
\section{Concurrencia y asignación a tareas}
\subsection{Mapa de tareas FreeRTOS}
Tabla con tareas, prioridad, stack, frecuencia.

\subsection{Sincronización y comunicación}
Colas, semáforos, exclusiones mutuas usadas.

% ---------- 6 Aspectos de datos ----------
\section{Modelo de datos}
\subsection{Estructuras persistentes}
Formato del snapshot de partida, layout EEPROM, checksum.

\subsection{Estructuras en RAM}
Buffers de audio, frame buffer, colas de eventos.

% ---------- 7 Manejo de errores y registro (logging) ----------
\section{Manejo de errores y logging}
Política de códigos de error, niveles de log, rutas de fallos críticos.

% ---------- 8 Requisitos de diseño no funcionales ----------
\section{Requisitos de diseño no funcionales}
Rendimiento (FPS), footprint de Flash/RAM, respuesta a interrupciones, requisitos de seguridad.

% ---------- 9 Referencias ----------
\section{Referencias}
\bibliographystyle{plain}
% \bibliography{biblio_design}

% ---------- Apéndices ----------
\appendix
\section{Apéndice A – Glosario completo de servicios}
Tabla resumen: Nombre, capa, descripción de 1 línea.

\section{Apéndice B – Diagramas adicionales}
Coloca aquí otros diagramas UML o listas de mensajes.

\end{document}
