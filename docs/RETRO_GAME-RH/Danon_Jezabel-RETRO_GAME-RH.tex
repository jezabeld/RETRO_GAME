% Created 2025-07-10
% Intended LaTeX compiler: pdflatex
\documentclass[11pt,a4paper]{article}

% ────────────── Paquetes básicos ──────────────
\usepackage[utf8]{inputenc}
\usepackage[T1]{fontenc}
\usepackage[spanish]{babel}
\usepackage{graphicx,grffile}
\usepackage{longtable,wrapfig,rotating}
\usepackage[normalem]{ulem}
\usepackage{amsmath,amssymb,textcomp}
\usepackage{tabularx,lastpage,enumitem}
\usepackage[table,xcdraw]{xcolor}
\usepackage[left=2cm,right=2.5cm,top=2.5cm,bottom=2.5cm]{geometry}
\usepackage{hyperref}

% ────────────── Comandos de metadatos ─────────
\newcommand*{\autor}[1]{\def\authorname{#1}}
\newcommand*{\titulo}[1]{\def\@title{#1}\def\ttitle{#1}}
\newcommand{\versionActual}{A}
\newcommand{\fechaA}{10/07/2025}
\newcommand{\docCode}{\normalsize RETRO\_GAME-RH versión \versionActual}
\newcommand{\fechaActual}{\fechaA}

\titulo{Videojuego portátil inspirado en consolas retro}
\autor{Lic. Jezabel Danon}

% ────────────── Encabezado / pie ──────────────
\usepackage{fancyhdr}
\fancyhf{}
\pagestyle{fancy}
\lhead{\includegraphics[width=3.5cm]{Figuras/logoFIUBA.pdf}}
\rhead{\normalsize\textbf{\@title}\\Especificación de requisitos de hardware\\\docCode}
\setlength{\headheight}{42pt}
\setlength{\footskip}{25pt}
\cfoot{\normalsize Página \thepage{} de \pageref{LastPage}}
\renewcommand{\headrulewidth}{1pt}
\renewcommand{\footrulewidth}{0.4pt}

% ────────────── Hipervínculos ─────────────────
\hypersetup{
  colorlinks=true,
  linkcolor=black,
  urlcolor=blue,
  pdftitle={\@title},
  pdfauthor={\authorname},
  pdflang={es}
}

% ────────────── Documento ─────────────────────
\begin{document}

% ----- Portada -----
\begin{titlepage}
  \centering
  \includegraphics[width=.7\textwidth]{Figuras/logoFIUBA.pdf}\par
  \vspace{1cm}
  {\Huge\textbf{\ttitle}}\par
  \vspace{1.5cm}
  {\Large\itshape Especificación de requisitos de hardware (ERH)\par}
  \vspace{3cm}
  \flushleft
  {\normalsize Autor:}\par
  {\Large \authorname\ (jezabel.danon@gmail.com)}\par
  \vspace{1.5cm}
  {\scshape\LARGE \fechaActual}\par
  {\scshape\LARGE Versión \versionActual}\par
  \vfill
  \centering
  \textit{Basado en la guía IEEE 1233-1998 y alineado con la ERS RETRO\_GAME-RS.}
\end{titlepage}

\clearpage
\tableofcontents
\clearpage

% ----- 1 Introducción -----
\section{Introducción}
\subsection{Propósito}
\begin{enumerate}
  \item Este documento representa una especificación de requerimientos de hardware para el sistema embebido \textit{\ttitle}. 
  \item Está dirigido a los desarrolladores que se ocupen del análisis, diseño e implementación del hardware, así como también a quienes desarrollen el testing, validaciones y/o verificaciones del mismo.
\end{enumerate}

\subsection{Ámbito}
Alcance del hardware cubierto (placa base, periféricos, conectividad).
\begin{enumerate}
  \item El hardware a desarrollar se compondrá por los siguiente elementos:
  \begin{itemize}
    \item Placa STM32 NUCLEO-F446RE
    \item Botones
    \item Joystick analógico
    \item Acelerómetro
    \item Pantalla
    \item Salida de audio
    \item Motor de vibraciones
  \end{itemize}
\end{enumerate}

\subsection{Definiciones, Acrónimos y Abreviaturas}

\subsection{Referencias}
Lista de datasheets, estándares (IEC, USB-IF, JEDEC), planos eléctricos preliminares.

\begin{enumerate}
  \item Especificaciones de requisitos de software: RETRO\_GAME-RS-vA.
  \item \href{https://drive.google.com/file/d/1C3vEYR8wME6EzlZVVC-gT2u86dwnoZA-/view?usp=sharing}{Plan de proyecto del trabajo práctico final} para la \textit{Carrera de Especialización en Sistemas Embebidos} (RETRO\_GAME-PP-v5). 
  \item Documento de diseño de software: RETRO\_GAME-DD.
\end{enumerate}

% ----- 2 Requisitos funcionales de hardware -----
\section{Requisitos funcionales}
\subsection{Entradas}
\begin{itemize}
  \item Botones físicos: 4 x tact switch, tiradores externos con pull-up interno.
  \item Joystick analógico de dos ejes, rango 0–3,3 V.
  \item Acelerómetro MEMS ±4 g, interfaz I\textsuperscript{2}C.
\end{itemize}

\subsection{Salidas}
\begin{itemize}
  \item Pantalla TFT 1,8" 128x160 px, controlador ST7735, interfaz SPI a 16 MHz.
  \item PWM audio vía filtro RC + amplificador mono 1 W.
  \item Motor vibración ERM + DRV2605L (I\textsuperscript{2}C).
\end{itemize}

% ----- 3 Requisitos de rendimiento -----
\section{Requisitos de rendimiento}
\begin{itemize}
  \item El bus SPI deberá soportar 16 MHz con \textless{} 1 % de error de reloj.
  \item La EEPROM externa (24LC512) deberá permitir al menos 400 kHz en I\textsuperscript{2}C.
  \item Duración mínima de batería: 2 h con brillo al 70 %.
\end{itemize}

% ----- 4 Requisitos ambientales -----
\section{Requisitos ambientales}
\begin{itemize}
  \item Temperatura de operación: 0 °C – 40 °C.
  \item Humedad relativa: 10 % – 90 %, sin condensación.
\end{itemize}

% ----- 5 Restricciones de diseño -----
\section{Restricciones de diseño}
\begin{itemize}
  \item Dimensiones máximas de la PCB: 90 mm × 60 mm.
  \item Todos los componentes deberán estar disponibles en paquetes SMD.
  \item Licencia de hardware: CERN OHL v2-P.
\end{itemize}

% ----- 6 Requisitos de seguridad y EMC -----
\section{Seguridad y compatibilidad electromagnética}
\begin{itemize}
  \item El dispositivo deberá cumplir con IEC 62368-1 para equipos audio/video.
  \item Emisiones radiadas CE: nivel Clase B (EN 55032).
\end{itemize}

% ----- 7 Plan de verificación -----
\section{Plan de verificación}
Breve descripción de las pruebas: continuidad de pines, consumo, señal SPI, EMC pre-scan.

% ----- 8 Apéndices -----
\appendix
\section{Lista de componentes preliminar}
Tabla BoM simplificada, precios de referencia, códigos de proveedor.

\end{document}
